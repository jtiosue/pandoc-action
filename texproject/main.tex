\newtheorem{theorem}{Theorem}
\newtheorem{corollary}[theorem]{Corollary}
\newtheorem{lemma}[theorem]{Lemma}
\theoremstyle{definition}
\newtheorem{definition}[theorem]{Definition}
\theoremstyle{remark}
\newtheorem{remark}{Remark}

\title{Title}
\author{Author}

\begin{definition}[right-angled triangles] \label{def:tri}
A \emph{right-angled triangle} is a triangle whose sides of length~\(a\), \(b\) and~\(c\), in some permutation of order, satisfies \(a^2+b^2=c^2\).
\end{definition}

\begin{lemma}
The triangle with sides of length~\(3\), \(4\) and~\(5\) is right-angled.
\end{lemma}

\begin{proof}
This lemma follows from \cref{def:tri} since \(3^2+4^2=9+16=25=5^2\).
\end{proof}

\begin{theorem}[Pythagorean triplets] \label{thm:py}
Triangles with sides of length \(a=p^2-q^2\), \(b=2pq\) and \(c=p^2+q^2\) are right-angled triangles.
\end{theorem}

\begin{remark}
These are all pretty interesting facts.
\end{remark}

\section{Another section}
Testing cite Ref \cite{temp}.

\bibliographystyle{plain} % We choose the "plain" reference style
\bibliography{references} % Entries are in the refs.bib file
